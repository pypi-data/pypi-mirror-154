\documentclass{article}
\usepackage{datatool}
\usepackage{graphicx}
\usepackage{datatool}

\begin{document}

\title{ANUGA applied to flooding in Towradgi}

\maketitle

\section{Introduction}
This model simulates the Towradgi Creek 17 August 1998 flood. It was largely
developed by Petar Milevski, with some edits by Gareth Davies and Steve
Roberts. Other flood studies which model this event can be found online from
2003 (Mike 11) and 2014 (TUFLOW) -- google `Towradgi Creek Flood Study' for
more information and to compare results across models. 

The current model (10/2014) only includes a few flow structures (four culverts
on the main channel), and ignores culvert blockage, which previous studies
found to be quite important. Therefore we consider it is in a draft state at
present, and could be refined by adding such features. Regardless it simulates
the main flow observations reasonably well.

\section{Results}

Figure~\ref{fig:spatial} shows the modelled peak flood depth, and compares it with flood
observations. Figure~\ref{fig:hist} shows a histogram of the model errors. The main area
with systematic discrepancy is just downstream of the Railway Bridge, and we
note that previous studies also had challenges in accurately modelling this
region in their calibration or validation events. 

\begin{figure}
\center
\includegraphics[width=\textwidth]{Spatial_Depth_and_Error.png}
\caption{Modelled peak depth, and the difference between the model and peak depth observations.}
\label{fig:spatial}
\end{figure}

\begin{figure}
\center
\includegraphics[width=\textwidth]{Error_peakstage.png}
\caption{Histogram of the difference between the model and the peak depth observations.}
\label{fig:hist}
\end{figure}

Figure~\ref{fig:pioneer} shows the modelled and measured stage at the Pioneer
River gauging station. The model gives good predictions of the timing of the
flows, but underpredicts the peak magnitude. This is likely to be related to
the lack of culvert blockage in the model. However we note that the 2014 study
also underpredicted the peak flow at this site, even when including culvert
blockage, and it also showed a greater delay in the timing of the peak. 

\begin{figure}
\center
\includegraphics[width=\textwidth]{Pioneer_Bridge_Stage.png}
\caption{Comparison of the model and the gauged stage at Pioneer bridge}
\label{fig:pioneer}
\end{figure}

\end{document}
